\chapter{Outputs}

Outputs are generic devices which receive packets. They are specialized
by a provided driver, which is responsible of processing
received packets (e.g. writing them into a file, distributing them over the 
network, to another device).

%
% SECTION How to write a processing engine
%
\section{How to write an output driver}

An output driver is built by three functions: \texttt{init} 
(described section \ref{sect:output_init}),
\texttt{destroy} (\ref{sect:output_destroy}) \& \texttt{transmit}
(\ref{sect:output_transmit}).

% Chain driver user functions list
\begin{itemize}
\item \texttt{init} is called when the output 
initialize. It is a good place if you want to allocate data or 
save options passed by the output layer.
\item \texttt{destroy} is called by the output when it 
is destroying. Free ressources allocated into \texttt{init} if
needef.
\item \texttt{transmit} is called on each packet received by 
the output. A pointer to the ressources that may have been
allocated into \texttt{init} is passed as an argument of this
call.
\end{itemize}

Once you you have written theses functions, you have to reference them
into a \texttt{output\_driver\_t} (\ref{sect:output_output_driver_t})
structure and pass it to the output driver register function to make
it available to the \textcolor{darkblue}{multitouchd} outputs.

%
% SECTION API
%
\section{API}

%
% SUBSECTION init
%
\subsection{init}
\label{sect:output_init}

% init function Figure
Your \texttt{init} function must follow this prototype:
\begin{lstlisting}[language=C,
caption=Output's init function prototype]
int
my_init_function
                (const mt_output_t * output,
                mt_output_driver_data_t ** driver_data, 
                const peach_hash_t * options);
\end{lstlisting}
Where:
\begin{itemize}
\item \texttt{options} is a hash table containing options passed 
to the output.
\item \texttt{driver\_data} is a pointer of pointer to a 
\texttt{mt\_output\_driver\_data\_t} structure you may define in 
your source file according to the figure \footnote{Pay attention 
to the underscore in front of the name of the structure: you must define 
\texttt{\_output\_driver\_data\_t} and not
\texttt{output\_driver\_data\_t}.}:
% _output_driver_data_t Figure
\begin{lstlisting}[language=C,
caption=\_mt\_output\_driver\_data\_t structure]
struct _mt_output_driver_data_t
{
    int whatever;
    [...]
};
\end{lstlisting}
If your input does \textbf{not} store data, you do \textbf{not} have to define 
the structure, either setting the value of \texttt{*driver\_data}.
\end{itemize}

%
% SUBSECTION destroy
%
\subsection{destroy}
\label{sect:output_destroy}
The \texttt{destroy} function is defined by:
% destroy function Figure
\begin{lstlisting}[language=C,
caption=Output's destroy function prototype]
int
my_destroy_function
            (const mt_output_t * output,
            mt_output_driver_data_t * driver_data);
\end{lstlisting}
Where:
% destroy funtion arguments list
\begin{itemize}
\item \texttt{output} is a pointer to the current input.
It could be used to retrieve its name.
\item \texttt{driver\_data} is a pointer to the area the 
\texttt{init} function may have allocated, or is undefined if 
the \texttt{init} function did not set its value.
\end{itemize}

%
% SUBSECTION transmit 
%
\subsection{transmit}
\label{sect:output_transmit}
The \texttt{run} function is defined by:
% transmit function Figure
\begin{lstlisting}[language=C,
caption=Output's transmit function prototype]
int 
my_transmit_function
                (const mt_output_t * output,
                mt_output_driver_data_t * driver_data,
                const char * from,
                const mt_packet_t * packet);
\end{lstlisting}
Where:
% transmit function arguments list
\begin{itemize}
\item \texttt{output} can be used to retreive the name of the output.
\item \texttt{driver\_data} is the pointer, defined or not, to the
area the \texttt{init} function may have allocated.
\item \texttt{from} is a \texttt{C} string which corresponds to the name
of the emitter of the packet.
\item \texttt{packet} is the packet to transmit. You do \textbf{not} have to
destroy it, its owner will.
\end{itemize}

This function is responsible of transmitting packets, it could take different
actions:
\begin{itemize}
\item Write packets into a file.
\item Send packets over the network.
\item Forward packets to an emulated system's device.
\item Print a console message (debugging purpose).
\end{itemize}

The \texttt{transmit} function is called on each packet the output receive. It
can perform blocking operation as when the output is destroying, the 
\texttt{SIGUSR1} signal is sent to the current thread to exit from a blocked
syscall.
Access to the \texttt{driver\_data} do \textbf{not} need to be serialized, 
because this area is \textbf{not} accessed by an other function until
\texttt{transmit} has exited.
%
% SUBSECTION output_layer_driver_t
%
\subsection{mt\_output\_driver\_t}
\label{sect:output_output_driver_t}
As usual, a structure is provided to pass to the output driver system the
three written functions : \texttt{init}
(line \ref{code:output_init_pointer}), 
\texttt{destroy} (line \ref{code:output_destroy_pointer}) \& 
\texttt{transmit} (line \ref{code:output_transmit_pointer}). Your functions 
must receive the same type of argument as the three function pointers 
described into the structure and return the same type:
% output_driver_t Figure
\begin{lstlisting}[escapeinside={//}{\^^M}, language=C,
caption=mt\_output\_driver\_t structure]
typedef struct 
{
    int 
    (*init)     //\label{code:output_init_pointer}
                (const mt_output_t * output,
                mt_output_driver_data_t ** driver_data, 
                const peach_hash_t * options);

    int 
    (*destroy)  //\label{code:output_destroy_pointer}
                (const mt_output_t * output,
                mt_output_driver_data_t * driver_data);

    int
    (*transmit)  //\label{code:output_transmit_pointer}
                (const mt_output_t * output,
                mt_output_driver_data_t * driver_data,
                const char * from,
                const mt_packet_t * packet);
}
mt_output_driver_t;
\end{lstlisting}

%
% SUBSECTION Registering & unregistering output driver
%
\subsection{Registering \& Unregistering driver}
\label{sect:output_driver_register}
\label{sect:output_driver_unregister}
To register an output driver you have to call, providing a name 
to reference the driver, the \texttt{mt\_output\_driver\_register} 
function. Hence you will be able to use this output driver in the 
\textcolor{darkblue}{Multitouch} library's functions related to output,
by giving this name in the arguments.
To unregister your driver, call the \texttt{mt\_output\_driver\_unregister} 
function as describes in the next figure:
\begin{lstlisting}[language=C,
caption=Output's driver management functions]
extern int
mt_output_driver_register
        (const char * name,
        const mt_output_driver_t * output_driver);

extern int
mt_output_driver_unregister
        (const char * name);
\end{lstlisting}

%
% SECTION Setup examples
%
\section{Setup examples}
TODO

%
% SECTION Example
%
\section{Complete module example}

This module defines an output driver which writes to a file (path is 
hardcoded) the packets it receives:
\begin{lstlisting}[language=C,
caption=Complete Output \& Module example]
#include "peach/module.h"
#include "peach/log.h"
#include "lib/output.h"

/* This path can passed as an argument of the 
 * output driver, see processing engine example.
 */
#define FILE_PATH "/tmp/multitouch_output.dump"

struct
{
    int file_descriptor;
}
_mt_output_driver_data_t;

static int 
output_example_driver_init
            (mt_output_driver_data_t ** driver_data, 
            const peach_hash_t * options)
{
    /* Allocate space to store the the output file 
     * descriptor 
     */
    *driver_data = malloc(sizeof(**driver_data));
    assert(*driver_data != 0);

    /* Try to open the output device */
    if (((*driver_data)->file_descriptor = 
                open(FILE_PATH, O_WRONLY)) < 0) {
        peach_log_err("Initialization error: could "
                    "not open file '%s': '%s'.\n", 
                    FILE_PATH, strerror(errno));

        goto clean;
    }

    return 0;

clean:
    free(*driver_data);

    return -1;
}

static int 
output_example_driver_destroy
            (mt_output_driver_data_t * driver_data)
{
    /* Close the output file */
    close(driver_data->file_descriptor);

    /* Free the allocated ressources */
    free(driver_data);

    return 0;
}

static int
output_example_driver_run
            (const mt_output_t * output,
            mt_output_driver_data_t * driver_data,
            const char * from,
            const mt_packet_t * packet)
{
    /* Dump the packet into the file */
    write(driver_data->output_descriptor,
                packet_serialize(packet),
                packet_get_length(packet));

    return 0;
}

/* This structure store the address of the three
 * driver's functions
 */
static mt_output_driver_t _example_driver =
{
    .init = output_example_driver_init,
    .destroy = output_example_driver_destroy,
    .run = output_example_driver_run
};

static int
example_module_init(void)
{
    /* Register our drivers by passing the address
     * of the mt_output_driver_t structure
     * filled with the three driver's functions 
     */
    if (mt_output_driver_register("example", 
                &_example_driver) != 0) 
        goto exit_with_failure;

    return 0;

exit_with_failure:
    return -1;
}

static int
example_module_exit(void)
{
    /* Unregister all drivers */
    mt_output_driver_unregister("example");
    
    return 0;
}

PEACH_MODULE_REGISTER("my_example_module", 1,
            example_module_init,
            example_module_exit)

\end{lstlisting}
