\chapter{Modules}
The \textcolor{darkblue}{Multitouch} library provide a way to inject code into 
an application during runtime: modules. They are mainly used to add inputs,
outputs and processing engine drivers, but could serve other purposes.
In addition, \textcolor{darkblue}{Multitouch}'s library checks module 
dependencies.

Modules must contains at least two functions, \texttt{init} 
(see \ref{sect:module_init}) \& \texttt{exit} (\ref{sect:module_exit}), 
which are used to handle the module, registered using a \texttt{C} 
pre-processing macro (\ref{sect:module_registration}).
When a module is loaded, the library ensures that its dependencies are 
respected (abort with an error message if \textbf{not}), then it calls the 
\textit{module's init function}. When the application does \textbf{not} need 
it anymore, the \textit{module's exit function} is called, and the module 
is detached.

Section \ref{sect:module_example} shows an example of a complete module.

%
% SECTION Init
%
\section{init}
\label{sect:module_init}
The \textit{init function} of your module is called when it is loaded. This 
function should register your module's content with an application.
It does \textbf{not} take argument but returns an error code, which equals 
\texttt{0} if the function succeed, else the application will be notified by 
the \textcolor{darkblue}{Multitouch} library that an error occured:
% module init Figure
\begin{lstlisting}[language=C, caption=Module's init function example]
/* static means the function is not exported directly to
 * the application, to prevent conflict 
 */
static int
example_module_init(void)
{
    /* Print a message in the console */
    peach_log_info("Example module loaded!\n");

    /* Exit with success */
    return 0; 
}
\end{lstlisting}

This function is \textbf{not} a \textit{drivers's init function} as 
described later in this document. For example, if your module declares two
different drivers, with their respective \textit{drivers's init function}.
Your source file should contain \textbf{two} \textit{drivers's init function} 
and \textbf{one} \textit{module's init function} (that you use to register your
two drivers).




%
% SECTION Exit
%
\section{exit}
\label{sect:module_exit}
The \textit{exit function} is called when the module is unloaded. Like the 
\textit{init function}, it does \textbf{not} take argument but returns an 
error code:
% module exit Figure
\begin{lstlisting}[language=C, caption=Module's exit function example]
static int
example_module_exit(void)
{
    peach_log_info("Example module unloaded!\n");

    return 0; 
}
\end{lstlisting}

This function is a good place to unregister the content of your module to
prevent the application from using it whereas it is \textbf{not} available 
anymore.

%
% SECTION Registration
%
\section{Registration}
\label{sect:module_registration}
A module needs some informations to be loadable:
\begin{itemize}
\item a \textit{name}.
\item a \textit{version number}.
\item its \textit{init function}'s address.
\item its \textit{exit function}'s address.
\item its \textit{depencies} with other modules.
\end{itemize}
Theses informations are saved by two \texttt{C} pre-processing macros:
\begin{itemize}
\item \texttt{PEACH\_MODULE\_REGISTER}.
\item \texttt{PEACH\_MODULE\_DEPENDENCIES}.
\end{itemize}
The name and the version will be used by the \textcolor{darkblue}{Multitouch}
library's dependencies resolver. The name and version of a dependencie must
match exactly the name and the version of a loaded module. The library does 
\textbf{not} assert that a loaded module, with a greater version, is compatible
with a request of a lower version module.

%
% SUBSECTION L_MODULE_REGISTER
%
\subsection{PEACH\_MODULE\_REGISTER}
This macros's arguments are:
\begin{enumerate}
\item the \textit{module name}, a \texttt{C} string.
\item the \textit{module version}, a \texttt{C} \texttt{uint\_64\_t}.
\item the \textit{module init function}, a function pointer (i.e function name).
\item the \textit{module exit function}, a function pointer (i.e function name).
\end{enumerate}

%
% SUBSECTION L_MODULE_DEPENDENCIES
%
\subsection{PEACH\_MODULE\_DEPENDENCIES}
This pre-processing macro takes a list of \texttt{PEACH\_MODULE\_NEED} macros 
as arguments. The macro \texttt{PEACH\_MODULE\_NEED} takes two arguments:
\begin{enumerate}
\item the \textit{requested module name}, a \texttt{C} string.
\item the \textit{requested module version}, a \texttt{C} \texttt{uint\_64\_t}.
\end{enumerate}
If it is \textbf{not} passed argument or if not presented, the library asserts
that the module does \textbf{not} have any dependencie!


%
% SECTION COMPILATION
%
\section{Compilation}
Your module must compiled with the same flags used to compile 
\textcolor{darkblue}{Multitouch} library (e.g. optimization, debug), and be 
linked with it. If you have the \textcolor{darkblue}{Multitouch} project's 
source tree, you just have to reference your module's source file in the 
\texttt{CMakeList.txt} file located in the root directory as shown in the 
following figure:
% configuration Figure
\begin{lstlisting}[language=make, caption=CMakeList.txt update]
# This block schedules the compilation of 
# module/my_module.c (your source file) to create 
# 'my_module'  object.
add_library(
    my_module
    MODULE

    module/my_module.c
)

# This block tells that our object should be linked 
# with the 'multitouch' library.
target_link_libraries(
    my_module

    multitouch
)

# And this block controls the name of the generated
# module.
set_target_properties(
    my_module
    PROPERTIES

    PREFIX ""
    SUFFIX ${MODULE_EXTENSION}
)
\end{lstlisting}

%
% SECTION Complete module example
%
\section{Complete module example}
\label{sect:module_example}
Here is a complete module example which does \texttt{not} contain
any driver:
\begin{lstlisting}[language=C, caption=Complete module example]
/* You can choose any name you desire for this function
 * as long as it is corresponding to the name passed in 
 * the PEACH_MODULE_REGISTER macro 
 */
static int
example_module_init(void)
{
    /* Here is where you may call the functions to 
     * register the drivers you have written into this 
     * source file 
     */
    return 0; 
}

static int
example_module_exit(void)
{
    /* And here is where you unregister them */
    return 0; 
}

/* Your module is named "example_module" version 00 of 
 * the 11 of January 2009 
 */
PEACH_MODULE_REGISTER("example_module", 2009011100,
    example_module_init,
    example_module_exit)

/* It needs two modules: "needed_module" version 
 * 2008122701 and "other_needed_module" version 
 * 2008121100 
 */
PEACH_MODULE_DEPENDENCIES(
    PEACH_MODULE_NEED("needed_module", 2008122701),
    PEACH_MODULE_NEED("other_needed_module", 2008121100))

\end{lstlisting}

