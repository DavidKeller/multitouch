\chapter{Processing engines}

Processing engines are designed to extra process packets. Processing
engines can be chained to permit multiple extra processings. You can
view them as protocols of a one-way network stack. When a packet
enters in this one-way stack, it pass through the first processing engine 
of the stack, and if this processing engine accepts the packet, it is 
processted to the next processing engine, and so on until it reachs the
last processing engine. In the end, the packet can be read on the top of 
the stack.
Processing engines are implemented using the 
\textcolor{darkblue}{Multitouch}'s \texttt{mt\_chain\_} api. This api
offers to the developper the possibility to inject custom processing engine 
logic, by writting a processing engine's driver.

%
% SECTION Where are they used
%
\section{Where are they used ?}

Each \texttt{inputs} \& \texttt{outputs} use processing engines to 
respectively perform post-processing on emited packets and 
pre-processing on received packets. 

%
% SECTION How to write a processing engine
%
\section{How to write a processing engine driver}

A processing engine (i.e. \texttt{mt\_chain\_layer\_t}) use a set of three 
functions written by the module developer: \texttt{init} 
(described section \ref{sect:pengine_init}), \texttt{destroy} 
(\ref{sect:pengine_destroy}) 
\& \texttt{process} (\ref{sect:pengine_process}).

% Chain driver user functions list
\begin{itemize}
\item \texttt{init} is called when the processing engine 
initialize. It is where you may want to allocate data or save options 
passed by the chain layer.
\item \texttt{destroy} is called by the processing engine when it 
is destroying. Free ressources allocated into \texttt{init} if
needed.
\item \texttt{process} is called each time a packet pass throught 
the processing engine. A pointer to the ressources \texttt{init} you may have
allocated into the \texttt{init} function is passed as an argument of this 
call. Also a function pointer is present in the arguments list: call it on
each packet you want to process to the upper processing engine.
\end{itemize}

Once you have written theses functions, you have to reference them
into a \texttt{mt\_chain\_layer\_driver\_t} 
(\ref{sect:pengine_chain_layer_driver_t}) structure and pass it to the chain 
layer driver register function to make it available to the 
\textcolor{darkblue}{Multitouch} inputs \& outputs.

You may write the processing engine driver function into a module and
call the \texttt{mt\_chain\_layer\_driver\_register} 
(\ref{sect:pengine_chain_driver_register}) into the module init 
function, as \texttt{mt\_chain\_layer\_driver\_unregister}
(\ref{sect:pengine_chain_driver_unregister}) into the module 
destroy function.

%
% SECTION API
%
\section{API}

%
% SUBSECTION init
%
\subsection{init}
\label{sect:pengine_init}

Your \texttt{init} function must respect the following prototype: 
% init function Figure
\begin{lstlisting}[language=C, 
caption=Processing engine's init function prototype]
int 
my_init_function
        (mt_chain_layer_driver_data_t ** driver_data, 
        const peach_hash_t * options);
\end{lstlisting}
Where: 
% init function arguments list 
\begin{itemize}
\item \texttt{options} is a hash table containing options passed 
to the processing engine.
\item \texttt{driver\_data} is a pointer of pointer to a 
\texttt{mt\_chain\_layer\_driver\_data\_t} structure you may define in 
your source file according to the figure \footnote{Pay attention 
to the underscore in front of the name of the structure: you must define 
\texttt{\_mt\_chain\_layer\_driver\_data\_t} and not
\texttt{mt\_chain\_layer\_driver\_data\_t}.}:
% _chain_layer_driver_data_t Figure
\begin{lstlisting}[language=C,
caption=Processing engine's \_mt\_chain\_layer\_driver\_data\_t structure]
struct _mt_chain_layer_driver_data_t
{
    int whatever;
    [...]
};
\end{lstlisting}
If your processing engine do \textbf{not} store data, you do \textbf{not} have 
to define the structure, either setting the value of \texttt{*driver\_data}.
\end{itemize}

If the \texttt{init} function return a value different of \texttt{0}, the
initialization of the processing engine is aborded, and 
\texttt{\textcolor{darkblue}{Multitouch}} will inform the application
with a proper failure message.


%
% SUBSECTION destroy
%
\subsection{destroy}
\label{sect:pengine_destroy}

The \texttt{destroy} function is defined by this prototype:
\begin{lstlisting}[language=C,
caption=Processing engine's exit function prototype]
int 
my_destroy_function
            (mt_chain_layer_driver_data_t * driver_data);
\end{lstlisting}
Where:
% destroy funtion arguments list
\begin{itemize}
\item \texttt{driver\_data} is a pointer to the area the 
\texttt{init} function may have allocated, or is undefined if 
the \texttt{init} function did not set its value.
\end{itemize}

It is responsible of freeing ressources the \texttt{init} function 
may have allocated and can do further actions e.g. logging, 
commiting results somewhere. 

% destroy function Figure

%
% SUBSECTION process 
%
\subsection{process}
\label{sect:pengine_process}

The \texttt{process} function is defined in the following figure:
% process function Figure
\begin{lstlisting}[language=C,
caption=Processing engine's process function prototype]
int 
my_process_function
            (mt_chain_layer_t * layer,
            mt_chain_layer_driver_data_t * driver_data,
            const char * from,
            const mt_packet_t * packet,
            mt_chain_driver_accept_t accept);
\end{lstlisting}
Where:
% process function arguments list
\begin{itemize}
\item \texttt{layer} is used by the \texttt{accept}
function pointer and has not any other utility at the moment.
\item \texttt{driver\_data} is the pointer, defined or not, to the
area the \texttt{init} function may have allocated.
\item \texttt{from} is the name of the sender of the packet, most of the
time it is an input name if not process unit decided to change it.
\item \texttt{packet} is the packet which is going through the
processing unit. It is a const packet, so if you want to modify it,
you will have to create a copy (use \texttt{mt\_packet\_copy}), to modify it 
and to send it instead. You are responsible of deleting packet you create
in this function. A good practice is to delete right after the call to the
\texttt{accept} function, because when this call returns, the packet
is not used anymore.
\item \texttt{accept} is a function pointer,  you call on the packet 
you want to send to the following processing engine. If you do not call it, 
the packet is dropped and the following processing engine is never 
called. You do not need to free the packet, it will be destroy by its owner.
% accept function pointer Figure
\begin{lstlisting}[language=C,
caption=Processing engine's accept function pointer]
typedef int
(*mt_chain_driver_accept_t)
        (mt_chain_layer_t * layer,
        const char * from,
        const mt_packet_t * packet);
\end{lstlisting}
\end{itemize}

It is called when a packet transits into the processing engine. This
function can:

% transits function actions list
\begin{itemize}
\item Forge a new packet and send it to the following processing
engine.
\item Drop the current packet, preventing it to reach the following
processing engine.
\item Log informations about the current packet.
\item ect..
\end{itemize}

Calls to the \texttt{process} function are serialized, so you can 
safely use the \texttt{driver\_data} area without synchronization 
mechanism (e.g. mutex).


%
% SUBSECTION chain_layer_driver_t
%
\subsection{mt\_chain\_layer\_driver\_t}
\label{sect:pengine_chain_layer_driver_t}

% chain_layer_driver_t Figure
\begin{lstlisting}[escapeinside={//}{\^^M}, language=C,
caption=mt\_chain\_layer\_driver\_t structure]
typedef struct 
{
    int 
    (*init)     //\label{code:pengine_init_pointer}
            (mt_chain_layer_driver_data_t ** driver_data, 
            const peach_hash_t * options);

    int 
    (*destroy)  //\label{code:pengine_destroy_pointer}
            (mt_chain_layer_driver_data_t * driver_data);

    int
    (*process)  //\label{code:pengine_process_pointer}
            (mt_chain_layer_t * layer,
            mt_chain_layer_driver_data_t * driver_data,
            const char * from,
            const mt_packet_t * packet,
            mt_chain_driver_accept_t accept);
}
mt_chain_layer_driver_t;
\end{lstlisting}
It is used to pass to the chain layer driver registering system the three
driver functions you will write : \texttt{init} 
(line \ref{code:pengine_init_pointer}), 
\texttt{destroy} (line \ref{code:pengine_destroy_pointer}) \& 
\texttt{process} (line \ref{code:pengine_process_pointer}). Your functions 
must receive the same type of argument as the three function pointers 
described into the structure and return the same type.

%
% SUBSECTION Registering & unregistering chain driver
%
\subsection{Registering \& Unregistering driver}
\label{sect:pengine_chain_driver_register}
\label{sect:pengine_chain_driver_unregister}

To register a processing engine driver you have to call, providing a name 
to reference the driver, the \texttt{mt\_chain\_layer\_driver\_register} 
function. Hence you will be able to use this
processing engine driver in the \textcolor{darkblue}{Multitouch} library's 
functions related to processing engine, by giving this name in the arguments.
To unregister, call the \texttt{mt\_chain\_layer\_driver\_unregister} 
function as describes in the next figure:
% chain driver register/unregister functions Figure
\begin{lstlisting}[language=C,
caption=Processing engine's driver registration functions]
extern int
mt_chain_layer_driver_register
        (const char * name,
        const mt_chain_layer_driver_t * layer_driver);

extern int
mt_chain_layer_driver_unregister
        (const char * name);
\end{lstlisting}

%
% SECTION Setup examples
%
\section{Setup examples}

The following setup is defined: two inputs input$^A$ \& input$^B$
and two outputs output$^A$ \& output$^B$.

We want to prevent packets from input$^A$ to be sent by output$^A$ and
transform the emission format of the outputs to xml.

We have to code two processing engines: \texttt{filter\_input} \& \texttt{xml}.
Then we add the \texttt{xml} processing engine on both inputs (or on both
outputs) and the \texttt{filter\_output} on the output$^A$ configuring it
to drop packets from emitter input$^A$.

%
% SUBSECTION Xml
%
\subsection{Xml}

This processing engine will transform an event into an xml event.

We decide to use the \texttt{xmllib2} for the transformation. We do not need
to initialize component of the library, but we have to save options passed by
the configuration system (e.g. the character encoding of the output).
We will:

\begin{enumerate}
\item Create a new module.
\item Write the processing engine \texttt{init} function which performs:
\begin{enumerate}
\item options saving.
\item \texttt{libxml} copyright printing into the log.
\end{enumerate}
\item Write the processing engine \texttt{destroy} function to free ressources
used by options saving.
\item Code the processing engine \texttt{process} function to:
\begin{enumerate}
\item initialize a xml writer to output into a char buffer.
\item open the packet, check its consistence and convert it using the 
xml writer.
\item destroy the xml writer.
\item create a new packet using the char buffer and transfert it to
the next processing engine using \texttt{accept} function.
\end{enumerate}
\item Register the processing engine into the module initialization function.
\item Load the module.
\item Configure outputs to use our processing engine using the dedicated 
function.
\end{enumerate}

%
% SUBSECTION Filter input
%
\subsection{Filter input}

This processing engine filter packets going through itself by dropping packets
according to the name of their emitter.

We just have to save options (i.e. the name of the emitter you want 
to drop packet) our functions will be quite simple. Here are the steps to 
write the processing engine:

\begin{enumerate}
\item Create a new module.
\item Write the processing engine \texttt{init} function which 
performs options saving.
\item Write the processing engine \texttt{destroy} function to free ressources
used by options saving.
\item Code the processing engine \texttt{process} function to check the name
of the imitter of the current packet, and drop it by not calling 
\texttt{accept} on it if the name is present in the list of emitters to 
silent passed into the configuration.
\item Register the processing engine into the module initialization function.
\item Load the module.
\item Configure $output^A$ to use our processing engine with $input^A$ passed
as the emitters to silent list.
\end{enumerate}

%
% SECTION Example
%
\section{Complete module example}

This module defines a processing engine driver which prints into console the 
emitter of each packet it receives:
\begin{lstlisting}[language=C,
caption=Complete Processing engine \& Module example]
#include "peach/module.h"
#include "peach/log.h"
#include "lib/chain.h"

struct _mt_chain_layer_driver_data_t
{
    uint64_t packets_count;
    FILE * output_descriptor;
};

static int 
chain_example_driver_init
            (mt_chain_layer_driver_data_t ** driver_data, 
            const peach_hash_t * options)
{
    const char * output;
    /* Allocate space to store the packets counter and
     * the output file descriptor 
     */
    *driver_data = malloc(sizeof(**driver_data));
    assert(*driver_data != 0);

    /* Set the packet counter to 0 */
    (*driver_data)->packets_count = 0;

    /* Retreive the path to the output device from the 
     * arguments 
     */
    if ((output = peach_hash_get(options, "output", 
                sizeof("output") - 1)) == 0) {
        peach_log_err("Initialization error: "
                    "'output' options is not passed"
                    " to the chain driver.\n");

        goto clean;
    }

    /* Try to open the output device */
    if (((*driver_data)->output_descriptor = 
                fopen(output, "w")) < 0) {
        peach_log_err("Initialization error: could "
                    "not open file '%s': '%s'.\n", 
                    output, strerror(errno));

        goto clean;
    }

    return 0;

clean:
    free(*driver_data);

    return -1;
}

static int 
chain_example_driver_destroy
            (mt_chain_layer_driver_data_t * driver_data)
{
    /* Close the output file */
    fclose(driver_data->output_descriptor);

    /* Free the allocated ressources */
    free(driver_data);

    return 0;
}

/* This function is called on each packet */
static int
chain_example_driver_process
            (mt_chain_layer_t * layer,
            mt_chain_layer_driver_data_t * driver_data,
            const char * from,
            const mt_packet_t * packet,
            mt_chain_driver_accept_t accept)
{
    driver_data->packets_count ++;

    /* Print the message */
    fprintf(driver_data->output_descriptor, 
                "Packet No '%u' from '%s'\n",
                driver_data->packets_count, from);

    /* Validate packet (send it to the upper 
     * processing engine). You have to pass the 'layer'
     * argument to the accept function however you can
     * pass a different 'from' string or a different 
     * 'packet'. 
     * This is usefull if you want to change the name 
     * of the emitter or forge a new packet.
     */
    return (*accept)(layer, from, packet);
}

/* This structure store the address of the three
 * driver's functions
 */
static mt_chain_layer_driver_t _example_driver =
{
    .init = chain_example_driver_init,
    .destroy = chain_example_driver_destroy,
    .process = chain_example_driver_process
};

static int
example_module_init(void)
{
    /* Register our drivers by passing the address
     * of the mt_chain_layer_driver_t structure
     * filled with the three driver's functions 
     */
    if (mt_chain_layer_driver_register("example", 
                &_example_driver) != 0) 
        goto exit_with_failure;

    return 0;

exit_with_failure:
    return -1;
}

static int
example_module_exit(void)
{
    /* Unregister all drivers */
    mt_chain_layer_driver_unregister("example");
    
    return 0;
}

PEACH_MODULE_REGISTER("my_example_module", 1,
            example_module_init,
            example_module_exit)

\end{lstlisting}
