\chapter{Inputs}

Inputs are generic devices which generate packets. They are specialized
by a provided driver, which is responsible of building packets 
(e.g. using datas from a device, a file, a random source).

%
% SECTION How to write an input driver
%
\section{How to write an input driver}

An input driver is built by three functions: \texttt{init} 
(described section \ref{sect:input_init}),
\texttt{destroy} (\ref{sect:input_destroy}) \& \texttt{run}
(\ref{sect:input_run}).

% Chain driver user functions list
\begin{itemize}
\item \texttt{init} is called when the input 
initialize. It is a good place if you want to allocate data or 
save options passed by the input layer.
\item \texttt{destroy} is called by the input when it 
is destroying. Free ressources allocated into \texttt{init} if
needef.
\item \texttt{run} which is called exactly one time, should generate 
multiple packets and send them using a function pointer passed in
arguments. A pointer to the ressources that may have been
allocated into \texttt{init} is passed as an argument of this
call.
\end{itemize}

Once you you have written theses functions, you have to reference them
into a \texttt{mt\_input\_driver\_t} (\ref{sect:input_input_driver_t})
structure and pass it to the input driver register function to make
it available to the \textcolor{darkblue}{Multitouch} inputs.

%
% SECTION API
%
\section{API}

%
% SUBSECTION init
%
\subsection{init}
\label{sect:input_init}

The \texttt{init} function is defined by::
\begin{lstlisting}[language=C,
caption=Input's init function prototype]
int 
my_init_function
                (const mt_input_t * input,
                mt_input_driver_data_t ** driver_data, 
                const peach_hash_t * options);
\end{lstlisting}
Where: 
% init function arguments list 
\begin{itemize}
\item \texttt{options} is a hash table containing options passed 
to the input.
\item \texttt{driver\_data} is a pointer of pointer to a 
\texttt{mt\_input\_driver\_data\_t} structure you may define in 
your source file according to the figure \footnote{Pay attention 
to the underscore in front of the name of the structure: you must define 
\texttt{\_input\_driver\_data\_t} and not
\texttt{input\_driver\_data\_t}.}:
% _chain_layer_driver_data_t Figure
\begin{lstlisting}[language=C,
caption=\_mt\_input\_driver\_data\_t structure]
struct _mt_input_driver_data_t
{
    int whatever;
    [...]
};
\end{lstlisting}
If your input does \textbf{not} store data, you do \textbf{not} have to define 
the structure, either setting the value of \texttt{*driver\_data}.
\end{itemize}

%
% SUBSECTION destroy
%
\subsection{destroy}
\label{sect:input_destroy}
The \texttt{destroy} function is defined by:
% destroy function Figure
\begin{lstlisting}[language=C,
caption=Input's destroy function prototype]
int 
my_destroy_function
                (const mt_input_t * input,
                mt_input_driver_data_t * driver_data);
\end{lstlisting}
Where:
% destroy funtion arguments list
\begin{itemize}
\item \texttt{input} is a pointer to the current input.
It could be used to retrieve its name.
\item \texttt{driver\_data} is a pointer to the area the 
\texttt{init} function may have allocated, or is undefined if 
the \texttt{init} function did not set its value.
\end{itemize}

It is responsible of freeing ressources the \texttt{init} function 
may have allocated and can do further actions. 

%
% SUBSECTION run 
%
\subsection{run}
\label{sect:input_run}
The \texttt{run} function is defined by:
% run function Figure
\begin{lstlisting}[language=C,
caption=Input's run function prototype]
int 
my_run_function
                (const mt_input_t * input,
                mt_input_driver_data_t * driver_data,
                mt_input_driver_commit_t driver_commit,
                mt_input_driver_must_stop_polling_t, 
                            must_stop_polling_on);
\end{lstlisting}
Where:
% transmit function arguments list
\begin{itemize}
\item \texttt{input} can be used to retreive the name of the input.
\item \texttt{driver\_data} is the pointer, defined or not, to the
area the \texttt{init} function may have allocated.
\item \texttt{driver\_commit} is a function pointer which must be called on 
the built packet, to forward it to the output(s). Here is its definition:
% commit function Figure
\begin{lstlisting}[language=C,
caption=Input's commit function pointer definition]
typedef int
(*mt_input_driver_commit_t)
            (const mt_input_t * input,
            const mt_packet_t * packet);
\end{lstlisting}
The \texttt{input} argument of this function is the \texttt{input} argument
of the \textit{input's run function}.
\item \texttt{must\_stop\_polling\_on} is a function pointer used to ensure
the polling must continue:
\begin{lstlisting}[language=C,
caption=Input's continuation test function pointer definition]
typedef int
(*mt_input_driver_must_stop_polling_t)
            (const mt_input_t * input);
\end{lstlisting}
The \texttt{input} argument of this function is the \texttt{input} argument
of the \textit{input's run function}.
\end{itemize}

This function is responsible of forging packets. It could build them by:
% transits function actions list
\begin{itemize}
\item Computing data from a system device opened in the \texttt{init} function.
\item Reading a file.
\item ect..
\end{itemize}

The \texttt{run} function is called only one time, and is exited only when
the input is destroying (i.e. \texttt{must\_stop\_polling\_on} returns \texttt{1}
and \texttt{SIGUSR1} has been sent to the current thread).
Blocking read could be performed into the \texttt{run} function, as the 
\texttt{SIGUSR1} will interrupt the read and permit exit.
Access to the \texttt{driver\_data} do \textbf{not} need to be serialized, 
because this area is \textbf{not} accessed by an other function until
\texttt{run} has exited.


%
% SUBSECTION input_layer_driver_t
%
\subsection{mt\_input\_driver\_t}
\label{sect:input_input_driver_t}
As usual, a structure is provided to pass to the input driver system the
three written functions : \texttt{init}
(line \ref{code:input_init_pointer}), 
\texttt{destroy} (line \ref{code:input_destroy_pointer}) \& 
\texttt{run} (line \ref{code:input_run_pointer}). Your functions 
must receive the same type of argument as the three function pointers 
described into the structure and return the same type:
% chain_layer_driver_t Figure
\begin{lstlisting}[escapeinside={//}{\^^M}, language=C,
caption=mt\_input\_driver\_t structure]
typedef struct 
{
    int 
    (*init)     //\label{code:input_init_pointer}
                (const mt_input_t * input,
                mt_input_driver_data_t ** driver_data, 
                const peach_hash_t * options);

    int 
    (*destroy)  //\label{code:input_destroy_pointer}
                (const mt_input_t * input,
                mt_input_driver_data_t * driver_data);

    int
    (*run)  //\label{code:input_run_pointer}
                (const mt_input_t * input,
                mt_input_driver_data_t * driver_data,
                mt_input_driver_commit_t driver_commit,
                mt_input_driver_must_stop_polling_t
                            must_stop_polling_on);
}
mt_chain_layer_driver_t;
\end{lstlisting}

%
% SUBSECTION Registering & unregistering input driver
%
\subsection{Registering \& Unregistering driver}
\label{sect:input_driver_register}
\label{sect:input_driver_unregister}

To register an input driver you have to call, providing a name 
to reference the driver, the \texttt{mt\_input\_driver\_register} 
function. Hence you will be able to use this input driver in the 
\textcolor{darkblue}{Multitouch} library's functions related to input,
by giving this name in the arguments.
To unregister your driver, call the \texttt{mt\_input\_driver\_unregister} 
function as describes in the next figure:
\begin{lstlisting}[language=C,
caption=Input's driver management functions]
extern int
mt_input_driver_register
        (const char * name,
        const mt_input_driver_t * input_driver);

extern int
mt_input_driver_unregister
        (const char * name);
\end{lstlisting}

%
% SECTION Setup examples
%
\section{Setup examples}
TODO

%
% SECTION Example
%
\section{Complete module example}

This module defines a null input driver which does \textbf{not} emit any packet:
\begin{lstlisting}[language=C,
caption=Complete Input \& Module example]
#include "peach/module.h"
#include "peach/log.h"
#include "lib/input.h"

static int 
input_example_driver_init
            (mt_input_driver_data_t ** driver_data, 
            const peach_hash_t * options)
{
    /* We do not need to store any data, so 'driver_data'
     * is kept untouched
     */
    return 0;
}

static int 
input_example_driver_destroy
            (mt_input_driver_data_t * driver_data)
{
    return 0;
}

static int
input_example_driver_run
            (const mt_input_t * input,
            mt_input_driver_data_t * driver_data,
            mt_input_driver_commit_t driver_commit,
            mt_input_driver_must_stop_polling_t 
                        must_stop_polling_on)
{
    /* Loop until the input is destroying 
     * (i.e must_stop_polling_on(input) return true.
     */
    while (!(*must_stop_polling_on)(input)) {

        /* Wait for the SIGUSR1 signal, that means the 
         * input is destroying.
         * Here is the place where the driver should 
         * poll data from a device using read() for 
         * example and call 'driver_commit' on the 
         * forged packet.
         */
        pause();
    }

    return 0;
}

/* This structure store the address of the three
 * driver's functions
 */
static mt_input_driver_t _example_driver =
{
    .init = input_example_driver_init,
    .destroy = input_example_driver_destroy,
    .run = input_example_driver_run
};

static int
example_module_init(void)
{
    /* Register our drivers by passing the address
     * of the mt_input_driver_t structure
     * filled with the three driver's functions 
     */
    if (mt_input_driver_register("example", 
                &_example_driver) != 0) 
        goto exit_with_failure;

    return 0;

exit_with_failure:
    return -1;
}

static int
example_module_exit(void)
{
    /* Unregister all drivers */
    mt_input_driver_unregister("example");
    
    return 0;
}

PEACH_MODULE_REGISTER("my_example_module", 1,
            example_module_init,
            example_module_exit)

\end{lstlisting}
